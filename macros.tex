% macros.tex
\ProvidesPackage{macros}

% General
\let\oldin\in
\let\oldcdot\cdot
\renewcommand{\in}{\ensuremath{\oldin}}
\renewcommand{\cdot}{\ensuremath{\oldcdot}}

\newcommand{\rtt}{\ensuremath{\sqrt{2}}}
\newcommand{\cbrt}[1]{\ensuremath{\sqrt[3]{#1}}}

\newcommand{\lt}{\left}
\newcommand{\rt}{\right}

\newcommand{\imp}{\ensuremath{\implies}}
\newcommand{\pmi}{\ensuremath{\impliedby}}
\renewcommand{\iff}{\ensuremath{\Longleftrightarrow}}

\newcommand{\qimp}{\ensuremath{\quad\implies\quad}}
\newcommand{\qpmi}{\ensuremath{\quad\impliedby\quad}}
\newcommand{\qiff}{\ensuremath{\quad\iff\quad}}

\newcommand{\tand}{\text{ and }}
\newcommand{\tor}{\text{ or }}
\newcommand{\tst}{\text{ such that }}

\newcommand{\qand}{\ensuremath{\quad\text{and}\quad}}
\newcommand{\qor}{\ensuremath{\quad\text{or}\quad}}
\newcommand{\qst}{\ensuremath{\quad\text{such that}\quad}}

\newcommand{\lprod}{\mathop{\prod}\limits}
\newcommand{\lsum}{\mathop{\sum}\limits}

\newcommand{\eps}{\epsilon} % epsilon
\newcommand{\veps}{\varepsilon} % epsilon variant
\newcommand{\vphi}{\varphi} % epsilon variant

\newcommand{\sq}{^2} % squared
\newcommand{\cb}{^3} % cubed

\newcommand{\nonzero}[1]{\ensuremath{#1_{\neq 0}}} % nonzero element
\newcommand{\sseq}{\ensuremath{\subseteq}} % subset or equal
\newcommand{\supeq}{\ensuremath{\subseteq}} % superset or equal

\newcommand{\boolT}{\bd{\tmo{True}}}  % bool True
\newcommand{\boolF}{\bd{\tmo{False}}} % bool False

\newcommand{\N}{\bbN} % Natural numbers
\newcommand{\Z}{\bbZ} % Integers
\newcommand{\Q}{\bbQ} % Rational numbers
\newcommand{\R}{\bbR} % Real numbers
\newcommand{\C}{\bbC} % Complex numbers

\newcommand{\F}{\bbF} % Finite field
\newcommand{\K}{\bbK} % General field

\newcommand{\re}{\operatorname{Re}} % Real part
\newcommand{\im}{\operatorname{Im}} % Imaginary part
\newcommand{\Arg}{\operatorname{Arg}} % Argument

\newcommand{\ol}{\overline}
\newcommand{\wt}{\widetilde}
\newcommand{\wh}{\widehat}
\renewcommand{\bar}[1]{\ol{#1}}

\newcommand{\Abar}{\ensuremath{\bar{A}}}
\newcommand{\Bbar}{\ensuremath{\bar{B}}} % cdef
\newcommand{\Gbar}{\ensuremath{\bar{G}}}
\newcommand{\Hbar}{\ensuremath{\bar{H}}} % ij
\newcommand{\Kbar}{\ensuremath{\bar{K}}}
\newcommand{\Lbar}{\ensuremath{\bar{L}}}
\newcommand{\Mbar}{\ensuremath{\bar{M}}} % nopqrstuvwxyz

\newcommand{\abar}{\ensuremath{\bar{a}}} % bcd
\newcommand{\ebar}{\ensuremath{\bar{e}}} % fghijklmnopqrstuv
\newcommand{\wbar}{\ensuremath{\bar{w}}}
\newcommand{\xbar}{\ensuremath{\bar{x}}} % yz

\newcommand{\ZZ}[1]{\ensuremath{\Z_{#1}}} % group of nonnegative integers mod input under addition
\newcommand{\Zn}{\ensuremath{\Z_n}}      % group of nonnegative integers mod n under addition
\newcommand{\Ztw}{\ensuremath{\Z_2}}     % group of nonnegative integers mod 2 under addition
\newcommand{\Zth}{\ensuremath{\Z_3}}     % group of nonnegative integers mod 3 under addition
\newcommand{\Zfo}{\ensuremath{\Z_4}}     % group of nonnegative integers mod 4 under addition
\newcommand{\Zfi}{\ensuremath{\Z_5}}     % group of nonnegative integers mod 5 under addition
\newcommand{\Zsi}{\ensuremath{\Z_6}}     % group of nonnegative integers mod 6 under addition
\newcommand{\Zse}{\ensuremath{\Z_7}}     % group of nonnegative integers mod 7 under addition
\newcommand{\Zei}{\ensuremath{\Z_8}}     % group of nonnegative integers mod 8 under addition
\newcommand{\Zni}{\ensuremath{\Z_9}}     % group of nonnegative integers mod 9 under addition
\newcommand{\Zte}{\ensuremath{\Z_{10}}}  % group of nonnegative integers mod 10 under addition

\newcommand{\FF}[1]{\ensuremath{\F_{#1}}} % finite field of n elements
\newcommand{\Ftw}{\ensuremath{\F_2}}     % finite field of 2 elements
\newcommand{\Fth}{\ensuremath{\F_3}}     % finite field of 3 elements
\newcommand{\Ffo}{\ensuremath{\F_4}}     % finite field of 4 elements
\newcommand{\Ffi}{\ensuremath{\F_5}}     % finite field of 5 elements
\newcommand{\Fsi}{\ensuremath{\F_6}}     % finite field of 6 elements
\newcommand{\Fse}{\ensuremath{\F_7}}     % finite field of 7 elements
\newcommand{\Fei}{\ensuremath{\F_8}}     % finite field of 8 elements
\newcommand{\Fni}{\ensuremath{\F_9}}     % finite field of 9 elements
\newcommand{\Fte}{\ensuremath{\F_{10}}}  % finite field of 10 elements

\newcommand{\half}{\frac{1}{2}}

\newcommand{\llist}[3]{\ensuremath{#1_{#2},\ldots,#1_{#3}}}

% Operators
\DeclareMathOperator{\sign}{sign} % signum
\DeclareMathOperator{\card}{card} % cardinality

%% Abstract Algebra
\DeclareMathOperator{\orb}{orb}   % orbit
\DeclareMathOperator{\stab}{stab} % stabilizer
\DeclareMathOperator{\chr}{char}  % characteristic
\DeclareMathOperator{\perm}{Perm} % permutation
\DeclareMathOperator{\Aut}{Aut}   % automorphism group
\DeclareMathOperator{\Inn}{Inn}   % inner automorphism group
\DeclareMathOperator{\Syl}{Syl}   % Sylow subgroup
\DeclareMathOperator{\Gal}{Gal}   % Galois group

%% Linear Algebra
\DeclareMathOperator{\proj}{proj}   % projection
\DeclareMathOperator{\Span}{span}   % span

%% Statistics
\DeclareMathOperator{\var}{\text{Var}} % variance


% Font variations
\newcommand{\nf}[1]{\textnormal{#1}}  % normal
\newcommand{\bd}[1]{\textbf{#1}}      % bold
\newcommand{\tmo}[1]{\texttt{#1}}     % monospace/typewriter
\renewcommand{\it}[1]{\emph{#1}}      % italics
\renewcommand{\ul}[1]{\uline{#1}}     % underline


% Add images to document
\newcommand{\img}[2]{\begin{figure}[H] % arg 1: image path, arg 2: width/scale
  \centering%
  \includegraphics[width=#2 \textwidth]{#1}%
  \end{figure}
}

% Homework solution environment
\newenvironment{solution}%
  {\begin{proof}[Solution]}%
  {\end{proof}}

% Subproof
\newenvironment{subproof}[1][Proof]{%
  \begin{proof}[#1] \renewcommand{\qedsymbol}{$\blacksquare$}}%
  {\end{proof}}


% Course specific

%% SPRING 2025
%%% CS 390 ATA "Advanced Topics on Algorithms"
\newcommand{\floor}[1]{\ensuremath{\lfloor #1 \rfloor}}
\newcommand{\ceil}[1]{\ensuremath{\lceil #1 \rceil}}
\newcommand{\doc}[1]{\noindent\it{\color{commentgreen} \texttt{/*}\ #1 \hfill \texttt{*/}}}
\newcommand{\defalgo}[2]{\noindent\uline{\tmo{#1(\ensuremath{#2}):}}}
\newcommand{\rcomm}[1]{\it{\color{commentgreen} \hfill \texttt{//} #1 }}
\newcommand{\lcomm}[1]{\it{\color{commentgreen} \texttt{//} #1 }}
\newcommand{\subcall}[2]{\tmo{\uline{#1(\ensuremath{#2})}}}

%%% MA 454 "Galois Theory Honors"
\newcommand{\mak}{\ensuremath{\mu_\alpha^K}} % minimum polynomial of alpha over K
\newcommand{\sfe}{splitting field extension}
\renewcommand{\sf}{splitting field}
\newcommand{\fdeg}[2]{\ensuremath{[ #1 : #2 ]}} % degree of field extension
\renewcommand{\ss}{\ensuremath{\subset}} % subset
\newcommand{\Fix}[2]{\ensuremath{\operatorname{Fix}_{#1}(#2)}} % Fixed field
\newcommand{\Tr}[1]{\ensuremath{\operatorname{Tr}(#1)}} % Trace
\newcommand{\Norm}[1]{\ensuremath{\operatorname{Norm}(#1)}} % Norm
\newcommand{\fp}{\ensuremath{\F_p}}  % finite field of p elements
\newcommand{\fq}{\ensuremath{\F_q}}  % finite field of q elements

%% FALL 2024
%%% CS 314 "Numerical Methods"
% \newcommand{\Prob}{\mathbb{P}}
% \newcommand{\Exep}{\mathbb{E}}

%%% MA 425 "Elements of Complex Analysis"
\newcommand{\len}[1]{\ensuremath{\lt\vert #1 \rt\vert}}

%%% MA 450 "Honors Abstract Algebra"
%%%% Shortcuts
\newcommand{\homo}{\text{homomorphism}} % hom
\newcommand{\inv}{\ensuremath{^{-1}}}   % inverse element
\newcommand{\tsgp}{\nf{\{\nf{id.}\}}}   % trivial subgroup
\newcommand{\iso}{\cong}       % isomorphism
\newcommand{\edp}{\oplus}      % external direct product
\newcommand{\idp}{\times}      % internal direct product
\newcommand{\U}[1]{\nf{U}(#1)} % group of units modulo n under multiplication

\newcommand{\contradiction}{\ensuremath{(\Rightarrow\!\Leftarrow)}} % contradiction

\newcommand{\order}[1]{\ensuremath{\lt\vert #1 \rt\vert}} % order of element/group
\newcommand{\cyc}[1]{\ensuremath{\langle #1 \rangle}}     % cyclic group
\renewcommand{\index}[2]{\ensuremath{[ #1 : #2 ]}}        % coset index

\newcommand{\divs}{\ensuremath{\mathrel{\vert}}}    % a divides k
\newcommand{\ndivs}{\ensuremath{\mathrel{\nmid}}}   % a does not divide k
\newcommand{\sgp}{\ensuremath{\mathrel{\leqslant}}} % subgroup
\newcommand{\psgp}{\ensuremath{\mathrel{<}}}        % proper subgroup
\newcommand{\ogp}{\ensuremath{\mathrel{\geqslant}}} % overgroup
\newcommand{\nsgp}{\ensuremath{\mathrel{\unlhd}}}   % normal subgroup
\newcommand{\nogp}{\ensuremath{\mathrel{\unrhd}}}   % normal overgroup
\newcommand{\pnsgp}{\ensuremath{\mathrel{\lhd}}}    % proper normal subgroup
\newcommand{\pnogp}{\ensuremath{\mathrel{\rhd}}}    % proper normal overgroup

\newcommand{\qg}[2]{\ensuremath{#1/#2}} % quotient group
\newcommand{\fr}[2]{\ensuremath{#1\,/\,#2}} % factor ring
\newcommand{\conj}[2]{\ensuremath{#1 #2 #1\inv}} % conjugation

\newcommand{\cl}[1]{\ensuremath{\operatorname{cl}( #1 )}} % conjugate class
\newcommand{\srg}{\ensuremath{\mathrel{\leqslant}}} % subring
\newcommand{\psrg}{\ensuremath{\mathrel{<}}}        % proper subring
% \newcommand{\tsrg}{\ensuremath{\{0\}}}              % trivial subring
\newcommand{\idl}{\ensuremath{\mathrel{\unlhd}}}    % ideal
\newcommand{\pidl}{\ensuremath{\mathrel{\lhd}}}     % proper ideal
\newcommand{\pid}[1]{\ensuremath{( #1 )}}           % principal ideal
\newcommand{\notidl}{\ntriangleleft}                % not ideal

%% SUMMER 2024
%%% STAT 350 "Introduction to Statistics"
\newcommand{\dint}[2]{\int_{#1}^{#2}}
\newcommand{\Dint}{\int_{-\infty}^{\infty}}

%% SPRING 2024
%%% MA 341 "Foundations of Analysis"
\newcommand{\abs}[1]{\ensuremath{\lt\vert #1 \rt\vert}}

%%% MA 35301 "Linear Algebra II"
\newcommand{\inner}[2]{\ensuremath{\lt\langle #1, #2 \rt\rangle}}
\newcommand{\norm}[1]{\ensuremath{\lt\vert\lt\vert #1 \rt\vert\rt\vert}}

%% FALL 2023
%%% MA 375 "Introduction to Discrete Mathematics"
\newcommand{\Mod}[1]{\ \mathrm{mod}\ #1}
