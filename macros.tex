% macros.tex
% \ProvidesPackage{macros}
\usepackage[normalem]{ulem}
% General
  \let\oldin\in
  \let\oldcdot\cdot
  \renewcommand{\in}{\ensuremath{\oldin}}
  \renewcommand{\cdot}{\ensuremath{\oldcdot}}

  \renewcommand{\emptyset}{\ensuremath{\varnothing}}

  \newcommand{\toinf}{\ensuremath{\to\infty}}

  \newcommand{\dd}[2]{\ensuremath{\frac{d#1}{d#2}}}
  \newcommand{\pdd}[2]{\ensuremath{\frac{\partial #1}{\partial #2}}}

  \newcommand{\rtt}{\ensuremath{\sqrt{2}}}

  \newcommand{\imp}{\ensuremath{\implies}}
  \newcommand{\pmi}{\ensuremath{\impliedby}}
  \renewcommand{\iff}{\ensuremath{\Longleftrightarrow}}

  \newcommand{\qimp}{\ensuremath{\quad\implies\quad}}
  \newcommand{\qpmi}{\ensuremath{\quad\impliedby\quad}}
  \newcommand{\qiff}{\ensuremath{\quad\iff\quad}}
  \newcommand{\qqquad}{\ensuremath{\qquad\qquad}}

  \newcommand{\tand}{\text{ and }}
  \newcommand{\tor}{\text{ or }}
  \newcommand{\tst}{\text{ s.t. }}

  \newcommand{\qand}{\quad\tand\quad}
  \newcommand{\qqand}{\quad\quad\tand\quad\quad}
  \newcommand{\qor}{\quad\tor\quad}
  \newcommand{\qqor}{\quad\quad\tor\quad\quad}
  \newcommand{\qst}{\quad\tst\quad}
  \newcommand{\qqst}{\quad\quad\tst\quad\quad}

  \newcommand{\limntoinf}{\ensuremath{\lim_{n\to\infty}}}
  \newcommand{\limtoinf}[1]{\ensuremath{\lim_{#1\to\infty}}}

  \newcommand{\floor}[1]{\ensuremath{\lfloor #1 \rfloor}}
  \newcommand{\ceil}[1]{\ensuremath{\lceil #1 \rceil}}

  \newcommand{\sq}{^2} % squared
  \newcommand{\cb}{^3} % cubed
  \newcommand{\half}{\frac{1}{2}}

  \newcommand{\nz}[1]{\ensuremath{#1_{\neq 0}}} % nonzero element
  \newcommand{\sseq}{\ensuremath{\subseteq}} % subset or equal
  \newcommand{\supeq}{\ensuremath{\supseteq}} % superset or equal

  \newcommand{\boolT}{\tmo{true}}  % boolean true
  \newcommand{\boolF}{\tmo{false}} % boolean false

  \newcommand{\N}{\ensuremath{\bbN}} % Natural numbers
  \newcommand{\Z}{\ensuremath{\bbZ}} % Integers
  \newcommand{\Q}{\ensuremath{\bbQ}} % Rational numbers
  \newcommand{\R}{\ensuremath{\bbR}} % Real numbers
  \newcommand{\C}{\ensuremath{\bbC}} % Complex numbers

  \newcommand{\Rk}{\ensuremath{\R^{k}}} % Euclidean k-space
  \newcommand{\Rn}{\ensuremath{\R^{n}}} % Euclidean n-space
  \newcommand{\Rm}{\ensuremath{\R^{m}}} % Euclidean m-space

  \newcommand{\F}{\ensuremath{\bbF}} % Finite field
  \newcommand{\K}{\ensuremath{\bbK}} % General field

  \newcommand{\ol}{\overline}
  \newcommand{\wt}{\widetilde}
  \newcommand{\wh}{\widehat}
  \renewcommand{\bar}[1]{\ol{#1}}

  \newcommand{\llist}[3]{\ensuremath{#1_{#2},\ldots,#1_{#3}}}
  \newcommand{\alist}[3]{\ensuremath{#1_{#2},\ldots,#1_{#3}}}
  \newcommand{\mlist}[3]{\ensuremath{#1_{#2},\ldots,#1_{#3}}}

  % \newcommand{\Abar}{\ensuremath{\bar{A}}}
  % \newcommand{\Bbar}{\ensuremath{\bar{B}}} % cdef
  % \newcommand{\Gbar}{\ensuremath{\bar{G}}}
  % \newcommand{\Hbar}{\ensuremath{\bar{H}}} % ij
  % \newcommand{\Kbar}{\ensuremath{\bar{K}}}
  % \newcommand{\Lbar}{\ensuremath{\bar{L}}}
  % \newcommand{\Mbar}{\ensuremath{\bar{M}}} % nopqrstuvwxyz

  % \newcommand{\abar}{\ensuremath{\bar{a}}} % bcdefghijklmnopqrstuv
  % \newcommand{\wbar}{\ensuremath{\bar{w}}}
  % \newcommand{\xbar}{\ensuremath{\bar{x}}}
  % \newcommand{\ybar}{\ensuremath{\bar{y}}}
  % \newcommand{\zbar}{\ensuremath{\bar{z}}}

  % \newcommand{\ZZ}[1]{\ensuremath{\Z_{#1}}} % Z mod input
  % \newcommand{\Zn}{\ensuremath{\Z_n}}      % Z mod n
  % \newcommand{\Ztw}{\ensuremath{\Z_2}}     % Z mod 2
  % \newcommand{\Zth}{\ensuremath{\Z_3}}     % Z mod 3
  % \newcommand{\Zfo}{\ensuremath{\Z_4}}     % Z mod 4
  % \newcommand{\Zfi}{\ensuremath{\Z_5}}     % Z mod 5
  % \newcommand{\Zsi}{\ensuremath{\Z_6}}     % Z mod 6
  % \newcommand{\Zse}{\ensuremath{\Z_7}}     % Z mod 7
  % \newcommand{\Zei}{\ensuremath{\Z_8}}     % Z mod 8
  % \newcommand{\Zni}{\ensuremath{\Z_9}}     % Z mod 9
  % \newcommand{\Zte}{\ensuremath{\Z_{10}}}  % Z mod 10

  % \newcommand{\FF}[1]{\ensuremath{\F_{#1}}} % finite field of input elements
  % \newcommand{\Fn}{\ensuremath{\F_n}} % finite field of input elements
  % \newcommand{\Ftw}{\ensuremath{\F_2}}      % finite field of 2 elements
  % \newcommand{\Fth}{\ensuremath{\F_3}}      % finite field of 3 elements
  % \newcommand{\Ffo}{\ensuremath{\F_4}}      % finite field of 4 elements
  % \newcommand{\Ffi}{\ensuremath{\F_5}}      % finite field of 5 elements
  % \newcommand{\Fsi}{\ensuremath{\F_6}}      % finite field of 6 elements
  % \newcommand{\Fse}{\ensuremath{\F_7}}      % finite field of 7 elements
  % \newcommand{\Fei}{\ensuremath{\F_8}}      % finite field of 8 elements
  % \newcommand{\Fni}{\ensuremath{\F_9}}      % finite field of 9 elements
  % \newcommand{\Fte}{\ensuremath{\F_{10}}}   % finite field of 10 elements


  % Greek
  \newcommand{\f}{\phi}
  \newcommand{\vf}{\varphi}

  \newcommand{\e}{\epsilon}
  \newcommand{\ve}{\varepsilon}

  % Operators
  \DeclareMathOperator{\sgn}{sgn} % signum
  \DeclareMathOperator{\card}{card} % cardinality

  % Topology
  \DeclareMathOperator{\Int}{Int}


  % Analysis
  \DeclareMathOperator{\re}{Re} % real part
  \DeclareMathOperator{\im}{Im} % imaginary part
  \DeclareMathOperator{\Arg}{Arg} % argument
  \DeclareMathOperator{\Log}{Log} % principal logarithm
  \DeclareMathOperator{\res}{Res} % residue

  % Algebra
  \DeclareMathOperator{\orb}{orb}   % orbit
  \DeclareMathOperator{\stab}{stab} % stabilizer
  \DeclareMathOperator{\chr}{char}  % characteristic
  \DeclareMathOperator{\perm}{Perm} % permutation
  \DeclareMathOperator{\Aut}{Aut}   % automorphism group
  \DeclareMathOperator{\Inn}{Inn}   % inner automorphism group
  \DeclareMathOperator{\Syl}{Syl}   % Sylow subgroup
  \DeclareMathOperator{\Gal}{Gal}   % Galois group
  \DeclareMathOperator{\Aff}{Aff}   % Affine group

  \DeclareMathOperator{\proj}{proj}   % projection
  \DeclareMathOperator{\Span}{span}   % span

  % Statistics
  \DeclareMathOperator{\var}{\text{Var}} % variance

% Font variation
  \newcommand{\nf}[1]{\textnormal{#1}}  % normal
  \newcommand{\bd}[1]{\textbf{#1}}      % bold
  \newcommand{\tmo}[1]{\texttt{#1}}     % monospace/typewriter
  \renewcommand{\it}[1]{\emph{#1}}      % italics
  \renewcommand{\ul}[1]{\uline{#1}}     % underline

% Add images to document
  \newcommand{\img}[2]{\begin{figure}[H] % $1 image path, $2 width/scale
    \centering%
    \includegraphics[width=#2 \textwidth]{#1}%
  \end{figure}
  }

% >>========|| Course specific  ||========<<
  % MA 440 "Honors Real Analysis"
    \newcommand{\ab}[1]{\ensuremath{\left\vert #1 \right\vert}} % absolute value
    \newcommand{\ip}[2]{\ensuremath{\left\langle #1,\!\ #2 \right\rangle}} % inner product
    \newcommand{\norm}[1]{\ensuremath{ \left\vert\:\!\!\left\vert #1 \right\vert\: \!\!\right\vert}} % vector norm
    \newcommand{\xnk}{\ensuremath{x_{n_k}}}
    \DeclareMathOperator{\osc}{osc}
    \DeclareMathOperator{\diam}{diam}

  % MA 525 "Introduction to Complex Analysis"

  % CS 390 ATA "Advanced Topics on Algorithms"
    % \newcommand{\defalgo}[2]{\noindent\uline{\tmo{#1(\ensuremath{#2}):}}}
    % \newcommand{\rcomm}[1]{\it{\color{commentgreen} \hfill \texttt{//} #1 }}
    % \newcommand{\lcomm}[1]{\it{\color{commentgreen} \texttt{//} #1 }}
    % \newcommand{\subcall}[2]{\tmo{\uline{#1(\ensuremath{#2})}}}

  % MA 454 "Galois Theory Honors"
    % \newcommand{\mak}{\ensuremath{\mu_\alpha^K}} % minimum polynomial of alpha over K
    % \newcommand{\fdeg}[2]{\ensuremath{\left[ #1 : #2 \right]}} % degree of field extension
    \newcommand{\Fix}{\ensuremath{\operatorname{Fix}}} % Fixed field
    \newcommand{\Tr}{\ensuremath{\operatorname{Tr}}} % Trace
    % \newcommand{\Norm}[1]{\ensuremath{\operatorname{Norm}(#1)}} % Norm
    \newcommand{\id}{\nf{Id.}} % Identity element

  % MA 425 "Elements of Complex Analysis"
    %

  % MA 450 "Honors Abstract Algebra"
    % \newcommand{\homo}{\text{homomorphism}} % hom
    \newcommand{\inv}{\ensuremath{^{-1}}}   % inverse element
    \newcommand{\tgp}{\nf{\{\id\}}}   % trivial subgroup
    \newcommand{\iso}{\cong}       % isomorphism
    \newcommand{\edp}{\oplus}      % external direct product

  \newcommand{\order}[1]{\ensuremath{\left\vert #1 \right\vert}} % order of element/group
  \newcommand{\cyc}[1]{\ensuremath{\langle #1 \rangle}}     % cyclic group
  \renewcommand{\index}[2]{\ensuremath{[ #1 : #2 ]}}        % coset index

  \newcommand{\divs}{\ensuremath{\mathrel{\vert}}}    % a divides k
  \newcommand{\ndivs}{\ensuremath{\mathrel{\nmid}}}   % a does not divide k

  \newcommand{\sgp}{\ensuremath{\mathrel{\leqslant}}} % subgroup
  \newcommand{\ogp}{\ensuremath{\mathrel{\geqslant}}} % overgroup
  \newcommand{\psgp}{\ensuremath{\mathrel{<}}}        % proper subgroup
  \newcommand{\nsgp}{\ensuremath{\mathrel{\unlhd}}}   % normal subgroup
  \newcommand{\nogp}{\ensuremath{\mathrel{\unrhd}}}   % normal overgroup
  \newcommand{\pnsgp}{\ensuremath{\mathrel{\lhd}}}    % proper normal subgroup
  \newcommand{\pnogp}{\ensuremath{\mathrel{\rhd}}}    % proper normal overgroup

  \newcommand{\qg}[2]{\ensuremath{#1/#2}} % quotient group
  \newcommand{\qr}[2]{\ensuremath{#1/#2}} % factor ring
  \newcommand{\conj}[2]{\ensuremath{#1 #2 #1\inv}} % conjugation

  \newcommand{\cl}[1]{\ensuremath{\operatorname{cl}( #1 )}} % conjugate class
  \newcommand{\srg}{\ensuremath{\mathrel{\leqslant}}} % subring
  \newcommand{\psrg}{\ensuremath{\mathrel{<}}}        % proper subring
  \newcommand{\idl}{\ensuremath{\mathrel{\unlhd}}}    % ideal
  \newcommand{\pidl}{\ensuremath{\mathrel{\lhd}}}     % proper ideal
  \newcommand{\pid}[1]{\ensuremath{( #1 )}}           % principal ideal
  \newcommand{\notidl}{\ntriangleleft}                % not ideal

  % STAT 350 "Introduction to Statistics"
  \newcommand{\dint}[2]{\int_{#1}^{#2}}
  \newcommand{\Dint}{\int_{-\infty}^{\infty}}

  % MA 375 "Introduction to Discrete Mathematics"
  \newcommand{\Mod}[1]{\ \mathrm{mod}\ #1}
