% macros.tex

% General
\let\oldin\in
\let\oldcdot\cdot
\renewcommand*{\in}{\ensuremath{\oldin}}
\renewcommand*{\cdot}{\ensuremath{\oldcdot}}

\newcommand*{\rtt}{\ensuremath{\sqrt{2}}}
\newcommand*{\cbrt}[1]{\ensuremath{\sqrt[3]{#1}}}

\newcommand*{\lt}{\left}
\newcommand*{\rt}{\right}

\newcommand*{\imp}{\ensuremath{\implies}}
\newcommand*{\pmi}{\ensuremath{\impliedby}}
\renewcommand*{\iff}{\ensuremath{\Longleftrightarrow}\ }
\newcommand*{\qimp}{\ensuremath{\quad\implies\quad}}
\newcommand*{\qpmi}{\ensuremath{\quad\impliedby\quad}}
\newcommand*{\qiff}{\ensuremath{\quad\iff\quad}}

\newcommand*{\tand}{\text{ and }}
\newcommand*{\tor}{\text{ or }}
\newcommand*{\tst}{\text{ such that }}

\newcommand*{\eps}{\epsilon} % epsilon
\newcommand*{\veps}{\varepsilon} % epsilon variant
\newcommand*{\vphi}{\varphi} % epsilon variant

\newcommand*{\sq}{^2} % squared
\newcommand*{\cb}{^3} % cubed

\newcommand*{\nonzero}[1]{\ensuremath{#1_{\neq 0}}} % nonzero element
\newcommand*{\sseq}{\ensuremath{\subseteq}} % subset or equal

\newcommand{\boolT}{\bd{True}}  % bool True
\newcommand{\boolF}{\bd{False}} % bool False

\newcommand{\N}{\bbN} % Natural numbers
\newcommand{\Z}{\bbZ} % Integers
\newcommand{\Q}{\bbQ} % Rational numbers
\newcommand{\R}{\bbR} % Real numbers
\newcommand{\C}{\bbC} % Complex numbers
\newcommand{\F}{\bbF} % Finite field
\newcommand{\K}{\bbK} % General field

\renewcommand{\bar}[1]{\overline{#1}}
\newcommand{\ol}{\overline}
\newcommand{\wt}{\widetilde}
\newcommand{\wh}{\widehat}
\providecommand{\half}{\frac{1}{2}}

% Operators
\DeclareMathOperator{\sign}{sign} % signum
\DeclareMathOperator{\card}{card} % cardinality

%% Group Theory
\DeclareMathOperator{\Aut}{Aut}   % automorphism group
\DeclareMathOperator{\Inn}{Inn}   % inner automorphism group
\DeclareMathOperator{\Syl}{Syl}   % Sylow subgroup
\DeclareMathOperator{\stab}{stab} % stabilizer
\DeclareMathOperator{\orb}{orb}   % orbit
\DeclareMathOperator{\perm}{Perm} % permutation
\DeclareMathOperator{\Gal}{Gal} % Galois group

%% Linear Algebra
\DeclareMathOperator{\proj}{proj}   % projection

% Font variations
\newcommand{\nf}[1]{\textnormal{#1}}  % normal
\renewcommand{\it}[1]{\emph{#1}}      % italics
\newcommand{\bd}[1]{\textbf{#1}}      % bold
\renewcommand{\ul}[1]{\uline{#1}}     % underline
\newcommand{\tmo}[1]{\texttt{#1}}     % monospace/typewriter

% Add images to document
\newcommand{\img}[2]{\begin{figure}[H] % arg 1: image path, arg 2: width/scale
  \centering%
  \includegraphics[width=#2 \textwidth]{#1}%
  \end{figure}
}

% Homework solution environment
\newenvironment{solution}%
  {\begin{proof}[Solution]}%
  {\end{proof}}

% Subproof
\newenvironment{subproof}[1][Proof]{%
  \begin{proof}[#1] \renewcommand{\qedsymbol}{$\blacksquare$}}%
  {\end{proof}}


% Course specific

%% SPRING 2025
%%% CS 390 ATA "Advanced Topics on Algorithms"
\newcommand{\floor}[1]{\ensuremath{\lfloor #1 \rfloor}}
\newcommand{\ceil}[1]{\ensuremath{\lceil #1 \rceil}}
\newcommand{\doc}[1]{\noindent\it{\color{commentgreen} \texttt{/*}\ #1 \hfill \texttt{*/}}}
\newcommand{\defalgo}[2]{\noindent\uline{\tmo{#1(\ensuremath{#2}):}}}
\newcommand{\rcomm}[1]{\it{\color{commentgreen} \hfill \texttt{//} #1 }}
\newcommand{\lcomm}[1]{\it{\color{commentgreen} \texttt{//} #1 }}
\newcommand{\subcall}[2]{\tmo{\uline{#1(\ensuremath{#2})}}}
\newcommand*{\Abar}{\ensuremath{\overline{A}}}
\newcommand*{\Bbar}{\ensuremath{\overline{B}}}


%%% MA 454 "Galois Theory Honors"
\newcommand*{\mak}{\ensuremath{\mu_\alpha^K}} % minimum polynomial of alpha over K
\newcommand{\FF}[1]{\ensuremath{\F_{#1}}} % finite field of n elements
\newcommand*{\Ftw}{\ensuremath{\F_2}}     % finite field of 2 elements
\newcommand*{\Fth}{\ensuremath{\F_3}}     % finite field of 3 elements
\newcommand*{\Ffo}{\ensuremath{\F_4}}     % finite field of 4 elements
\newcommand*{\Ffi}{\ensuremath{\F_5}}     % finite field of 5 elements
\newcommand*{\Fsi}{\ensuremath{\F_6}}     % finite field of 6 elements
\newcommand*{\Fse}{\ensuremath{\F_7}}     % finite field of 7 elements
\newcommand*{\Fei}{\ensuremath{\F_8}}     % finite field of 8 elements
\newcommand*{\Fni}{\ensuremath{\F_9}}     % finite field of 9 elements
\newcommand*{\Fte}{\ensuremath{\F_{10}}}  % finite field of 10 elements
\newcommand*{\sfe}{splitting field extension}
\renewcommand*{\sf}{splitting field}
\newcommand*{\fdeg}[2]{\ensuremath{[ #1 : #2 ]}} % degree of field extension
\renewcommand*{\ss}{\ensuremath{\subset}} % subset
\newcommand*{\Lbar}{\ensuremath{\overline{L}}}
\newcommand*{\Mbar}{\ensuremath{\overline{M}}}
\newcommand{\Fix}[1]{\ensuremath{\operatorname{Fix}(#1)}} % Fixed field
\newcommand{\Tr}[1]{\ensuremath{\operatorname{Tr}(#1)}} % Trace
\newcommand{\Norm}[1]{\ensuremath{\operatorname{Norm}(#1)}} % Norm
\newcommand*{\Fp}{\ensuremath{\F_p}}  % finite field of p elements
\newcommand*{\Fq}{\ensuremath{\F_q}}  % finite field of q elements

%% FALL 2024
%%% CS 314 "Numerical Methods"
\newcommand{\Prob}{\mathbb{P}}
\newcommand{\Exep}{\mathbb{E}}
\newcommand{\asmp}{\textbf{Assumptions: }}
\DeclareMathOperator{\var}{\text{Var}} % variance

%%% MA 425 "Elements of Complex Analysis"
\newcommand{\len}[1]{\ensuremath{\lt\vert #1 \rt\vert}}
\newcommand{\re}{\operatorname{Re}}
\newcommand{\im}{\operatorname{Im}}
\newcommand{\Arg}{\operatorname{Arg}}
\newcommand{\wbar}{\overline{w}}

%%% MA 450 "Honors Abstract Algebra"
\newcommand{\divs}{\ensuremath{\,\big\vert\,}} % a divides k <=> a \divs k
\newcommand*{\ndivs}{\ensuremath{\,\nmid\,}} % a does not divide k <=> a \ndivs k
\newcommand*{\suth}{\text{such that\ }}
\newcommand*{\homo}{\text{homomorphism}}
\newcommand{\inv}{\ensuremath{^{-1}}}
\newcommand{\order}[1]{\ensuremath{\lt\vert #1 \rt\vert}} % order of element/group
\newcommand*{\sgp}{\ensuremath{\leq}} % subgroup
\newcommand*{\psgp}{\ensuremath{<}} % proper subgroup
\newcommand*{\tsgp}{\ensuremath{\{e\}}} % trivial subgroup
\newcommand{\cyc}[1]{\ensuremath{\langle #1 \rangle}} % cyclic group
\newcommand*{\nsgp}{\trianglelefteq} % normal subgroup
\newcommand*{\pnsgp}{\triangleleft} % proper normal subgroup
\newcommand*{\iso}{\cong} % isomorphism
\renewcommand*{\index}[2]{\ensuremath{[ #1 : #2 ]}} % coset index
\newcommand*{\edp}{\oplus} % external direct product
\newcommand*{\idp}{\times} % internal direct product
\newcommand*{\abar}{\ensuremath{\overline{a}}}
\newcommand*{\ebar}{\ensuremath{\overline{e}}}
\newcommand*{\xbar}{\ensuremath{\bar{x}}}
\newcommand*{\Gbar}{\ensuremath{\overline{G}}}
\newcommand*{\Hbar}{\ensuremath{\overline{H}}}
\newcommand*{\Kbar}{\ensuremath{\overline{K}}}
\newcommand*{\contradiction}{\ensuremath{(\Rightarrow\!\Leftarrow)}} % contradiction
\newcommand{\U}[1]{U(#1)} % group of units modulo n under multiplication
\newcommand{\ZZ}[1]{\ensuremath{\Z_{#1}}} % group of nonnegative integers < input under addition
\newcommand*{\Zn}{\ensuremath{\Z_n}}      % group of nonnegative integers < n under addition
\newcommand*{\Ztw}{\ensuremath{\Z_2}}     % group of nonnegative integers < 2 under addition
\newcommand*{\Zth}{\ensuremath{\Z_3}}     % group of nonnegative integers < 3 under addition
\newcommand*{\Zfo}{\ensuremath{\Z_4}}     % group of nonnegative integers < 4 under addition
\newcommand*{\Zfi}{\ensuremath{\Z_5}}     % group of nonnegative integers < 5 under addition
\newcommand*{\Zsi}{\ensuremath{\Z_6}}     % group of nonnegative integers < 6 under addition
\newcommand*{\Zse}{\ensuremath{\Z_7}}     % group of nonnegative integers < 7 under addition
\newcommand*{\Zei}{\ensuremath{\Z_8}}     % group of nonnegative integers < 8 under addition
\newcommand*{\Zni}{\ensuremath{\Z_9}}     % group of nonnegative integers < 9 under addition
\newcommand*{\Zte}{\ensuremath{\Z_{10}}}     % group of nonnegative integers less than 9 under addition
% \newcommand{\orb}[2]{\ensuremath{\operatorname{orb}_{#1}(#2)}} % orbit
% \newcommand{\stab}[2]{\ensuremath{\operatorname{stab}_{#1}(#2)}} % stabilizer
\newcommand{\qg}[2]{\ensuremath{#1/#2}} % quotient group
% \newcommand*{\conj}[2]{\ensuremath{#1 #2 #1\inv}} % conjugate
% \newcommand{\cl}[1]{\ensuremath{\operatorname{cl}( #1 )}} % conjugate class
\renewcommand*{\char}[1]{\ensuremath{\operatorname{char}(#1)}} % characteristic
\newcommand{\defmap}[5]{\ensuremath{#1:\begin{matrix}
    #2 \to #3 \\
    #4 \mapsto #5
\end{matrix}}}
\newcommand*{\srg}{\ensuremath{\leq}} % subring
\newcommand*{\psrg}{\ensuremath{<}} % proper subring
\newcommand*{\tsrg}{\ensuremath{\{0\}}} % trivial subring
\newcommand*{\idl}{\trianglelefteq} % ideal
\newcommand*{\pidl}{\triangleleft} % ideal
% \newcommand{\pid}[1]{\ensuremath{\langle #1 \rangle}} % principal ideal
\newcommand*{\notidl}{\ntriangleleft} % not ideal
\newcommand{\fr}[2]{\ensuremath{#1\,/\,#2}} % factor ring

%% SUMMER 2024
%%% STAT 350 "Introduction to Statistics"
\newcommand{\dint}[2]{\int_{#1}^{#2}}
\newcommand{\Dint}{\int_{-\infty}^{\infty}}

%% SPRING 2024
%%% MA 341 "Foundations of Analysis"
\newcommand{\abs}[1]{\ensuremath{\lt\vert #1 \rt\vert}}

%%% MA 35301 "Linear Algebra II"
\newcommand{\inner}[2]{\ensuremath{\lt\langle #1, #2 \rt\rangle}}
\newcommand{\norm}[1]{\ensuremath{\lt\vert\lt\vert #1 \rt\vert\rt\vert}}
\newcommand{\llist}[3]{\ensuremath{#1_{#2},\ldots,#1_{#3}}}

%% FALL 2023
%%% MA 375 "Introduction to Discrete Mathematics"
\newcommand{\Mod}[1]{\ \mathrm{mod}\ #1}
