% macros.tex

% General
\let\oldin\in
\let\oldcdot\cdot
\renewcommand*{\in}{\ensuremath{\oldin}}
\renewcommand*{\cdot}{\ensuremath{\oldcdot}}
\newcommand*{\rtt}{\ensuremath{\sqrt{2}}}
\newcommand*{\lt}{\left}
\newcommand*{\rt}{\right}
\newcommand*{\st}{\ensuremath{\ \vert \ }} % such that (set builder)
\newcommand*{\bigst}{\ensuremath{\ \big|\ }} % big such that (set builder)
\newcommand*{\biggst}{\ensuremath{\ \bigg|\ }} % bigg such that (set builder)
\newcommand*{\Bigst}{\ensuremath{\ \Big|\ }} % bigg such that (set builder)
\newcommand*{\imp}{\ensuremath{\implies}}
\newcommand*{\pmi}{\ensuremath{\impliedby}}
\renewcommand*{\iff}{\ensuremath{\Longleftrightarrow\ }}
\newcommand*{\qimp}{\ensuremath{\quad\implies\quad}}
\newcommand*{\qpmi}{\ensuremath{\quad\impliedby\quad}}
\newcommand*{\qiff}{\ensuremath{\quad\iff\quad}}
\newcommand*{\tand}{\text{ and }}
\newcommand*{\tor}{\text{ or }}
\newcommand*{\tst}{\text{ such that }}
\newcommand*{\eps}{\epsilon}
\newcommand*{\veps}{\varepsilon}
\newcommand*{\sq}{^2}
\newcommand*{\cb}{^3}
\newcommand*{\nonzero}[1]{\ensuremath{#1_{\neq 0}}}
\newcommand*{\sseq}{\ensuremath{\subseteq}}

% Font
\newcommand{\ital}[1]{\emph{#1}} % italics
\newcommand{\bd}[1]{\textbf{#1}} % bold
\newcommand{\ul}[1]{\underline{#1}} % underline

% Graphics
\newcommand{\img}[2]{\begin{figure}[H] % insert image file
  \centering%
  \includegraphics[height=#2 \textwidth]{#1}%
  \end{figure}}

% For STAT 350 "Introduction to Statistics"
\newcommand{\dint}[2]{\int_{#1}^{#2}}
\newcommand{\Dint}{\int_{-\infty}^{\infty}}

% MA 375 "Introduction to Discrete Mathematics"
\newcommand{\Mod}[1]{\ \mathrm{mod}\ #1}

% MA 35301 "Linear Algebra II"
\newcommand{\inner}[2]{\ensuremath{\lt\langle #1, #2 \rt\rangle}}
\newcommand{\norm}[1]{\ensuremath{\lt\vert\lt\vert #1 \rt\vert\rt\vert}}

% MA 341 "Foundations of Analysis"
\newcommand{\abs}[1]{\ensuremath{\lt\vert #1 \rt\vert}}

% CS 314 "Numerical Methods"
\newcommand{\Prob}{\mathbb{P}}
\newcommand{\Exep}{\mathbb{E}}
\newcommand{\asmp}{\textbf{Assumptions: }}
\DeclareMathOperator{\var}{\text{Var}}
% \newcommand{\nd}{\mathcal{N}}

% MA 425 "Elements of Complex Analysis"
% \newcommand{\len}[1]{\ensuremath{\lt\vert #1 \rt\vert}}
% \newcommand{\re}{\operatorname{Re}}
% \newcommand{\im}{\operatorname{Im}}
% \newcommand{\Arg}{\operatorname{Arg}}
% \newcommand{\wbar}{\overline{w}}
\renewcommand{\bar}[1]{\overline{#1}}

% MA 450 "Honors Abstract Algebra"
%% GROUPS
\newcommand*{\divs}{\ensuremath{\,\big\vert\,}} % a divides k <=> a \divs k
\newcommand*{\ndivs}{\ensuremath{\,\nmid\,}} % a does not divide k <=> a \ndivs k
% \newenvironment{claim}[1]{\par\noindent\underline{Claim:}\space#1}{}
\newcommand*{\suth}{\text{such that}}
\newcommand*{\homo}{\text{homomorphism}}
\newcommand{\inv}{\ensuremath{^{-1}}}
\newcommand{\order}[1]{\ensuremath{\lt\vert #1 \rt\vert}} % order of element/group

\newcommand*{\sgp}{\ensuremath{\leq}} % subgroup
\newcommand*{\psgp}{\ensuremath{<}} % proper subgroup
\newcommand*{\tsgp}{\ensuremath{\{e\}}} % trivial subgroup
\newcommand{\cyc}[1]{\ensuremath{\langle #1 \rangle}} % cyclic group
\newcommand*{\nsgp}{\vartriangleleft} % normal subgroup

\newcommand*{\iso}{\cong} % isomorphism

\renewcommand*{\index}[2]{\ensuremath{[ #1 : #2 ]}} % coset index

\newcommand*{\edp}{\oplus} % external direct product
\newcommand*{\idp}{\times} % internal direct product
\newcommand*{\kerphi}{\ker\phi} % kernel of the mapping \phi

\newcommand*{\abar}{\ensuremath{\overline{a}}}
\newcommand*{\ebar}{\ensuremath{\overline{e}}}
\newcommand*{\xbar}{\ensuremath{\bar{x}}}

\newcommand*{\Gbar}{\ensuremath{\overline{G}}}
\newcommand*{\Hbar}{\ensuremath{\overline{H}}}
\newcommand*{\Kbar}{\ensuremath{\overline{K}}}

\newcommand*{\contradiction}{\ensuremath{(\Rightarrow\!\Leftarrow)}} % contradiction

\newcommand{\U}[1]{U(#1)} % group of units modulo n under multiplication

\newcommand{\ZZ}[1]{\ensuremath{\Z_{#1}}} % group of nonnegative integers less than [INPUT] under addition
\newcommand*{\Zn}{\ensuremath{\Z_n}}    % group of nonnegative integers less than n under addition
\newcommand*{\Ztw}{\ensuremath{\Z_2}}     % group of nonnegative integers less than 2 under addition
\newcommand*{\Zth}{\ensuremath{\Z_3}}     % group of nonnegative integers less than 3 under addition
\newcommand*{\Zfo}{\ensuremath{\Z_4}}     % group of nonnegative integers less than 4 under addition
\newcommand*{\Zfi}{\ensuremath{\Z_5}}     % group of nonnegative integers less than 5 under addition
\newcommand*{\Zsi}{\ensuremath{\Z_6}}     % group of nonnegative integers less than 6 under addition
\newcommand*{\Zse}{\ensuremath{\Z_7}}     % group of nonnegative integers less than 7 under addition
\newcommand*{\Zei}{\ensuremath{\Z_8}}     % group of nonnegative integers less than 8 under addition
\newcommand*{\Zni}{\ensuremath{\Z_9}}     % group of nonnegative integers less than 9 under addition
\newcommand*{\Zte}{\ensuremath{\Z_{10}}}     % group of nonnegative integers less than 9 under addition

\newcommand{\orb}[2]{\ensuremath{\operatorname{orb}_{#1}(#2)}} % orbit
\newcommand{\stab}[2]{\ensuremath{\operatorname{stab}_{#1}(#2)}} % stabilizer

\newcommand{\qg}[2]{\ensuremath{#1\,/\,#2}} % quotient group

\newcommand*{\conj}[2]{\ensuremath{#1 #2 #1\inv}} % conjugate
\newcommand{\cl}[1]{\ensuremath{\operatorname{cl}( #1 )}} % conjugate class

%% RINGS
\renewcommand*{\char}[1]{\ensuremath{\operatorname{char}(#1)}} % characteristic

\newcommand{\defmap}[5]{\ensuremath{#1:\begin{matrix}
    #2 \to #3 \\
    #4 \mapsto #5
\end{matrix}}}

\newcommand*{\srg}{\ensuremath{\leq}} % subring
\newcommand*{\psrg}{\ensuremath{<}} % proper subring
\newcommand*{\tsrg}{\ensuremath{\{0\}}} % trivial subring
\newcommand*{\idl}{\vartriangleleft} % ideal
\newcommand{\pidl}[1]{\ensuremath{\langle #1 \rangle}} % principal ideal
\newcommand*{\notidl}{\ntriangleleft} % not ideal

\newcommand{\fr}[2]{\ensuremath{#1\,/\,#2}} % factor ring

% From Math 55 and Math 145 at Harvard
\newenvironment{subproof}[1][Proof]{%
\begin{proof}[#1] \renewcommand{\qedsymbol}{$\blacksquare$}}%
{\end{proof}}


\newcommand{\ol}{\overline}
% \newcommand{\ul}{\underline}
\newcommand{\wt}{\widetilde}
\newcommand{\wh}{\widehat}
\newcommand{\vocab}[1]{\textbf{\color{blue} #1}}
% \newcommand{\exercise}[1]{\textbf{\color{red} #1}}
\providecommand{\half}{\frac{1}{2}}
\newcommand{\dang}{\measuredangle} %% Directed angle
\newcommand{\ray}[1]{\overrightarrow{#1}}
\newcommand{\seg}[1]{\overline{#1}}
\newcommand{\arc}[1]{\wideparen{#1}}
\DeclareMathOperator{\cis}{cis}
\DeclareMathOperator*{\lcm}{lcm}
\DeclareMathOperator*{\argmin}{arg min}
\DeclareMathOperator*{\argmax}{arg max}
\newcommand{\cycsum}{\sum_{\mathrm{cyc}}}
\newcommand{\symsum}{\sum_{\mathrm{sym}}}
\newcommand{\cycprod}{\prod_{\mathrm{cyc}}}
\newcommand{\symprod}{\prod_{\mathrm{sym}}}
\newcommand{\Qed}{\begin{flushright}\qed\end{flushright}}
\newcommand{\parinn}{\setlength{\parindent}{1cm}}
\newcommand{\parinf}{\setlength{\parindent}{0cm}}
% \newcommand{\norm}{\|\cdot \|}
\newcommand{\inorm}{\norm_{\infty}}
\newcommand{\opensets}{\{V_{\alpha}\}_{\alpha\in I}}
\newcommand{\oset}{V_{\alpha}}
\newcommand{\opset}[1]{V_{\alpha_{#1}}}
\newcommand{\lub}{\text{lub}}
\newcommand{\del}[2]{\frac{\partial #1}{\partial #2}}
\newcommand{\Del}[3]{\frac{\partial^{#1} #2}{\partial^{#1} #3}}
\newcommand{\deld}[2]{\dfrac{\partial #1}{\partial #2}}
\newcommand{\Deld}[3]{\dfrac{\partial^{#1} #2}{\partial^{#1} #3}}
\newcommand{\lm}{\lambda}
\newcommand{\uin}{\mathbin{\rotatebox[origin=c]{90}{$\in$}}}
\newcommand{\usubset}{\mathbin{\rotatebox[origin=c]{90}{$\subset$}}}
% \newcommand{\lt}{\left}
% \newcommand{\rt}{\right}
\newcommand{\bs}[1]{\boldsymbol{#1}}
\newcommand{\exs}{\exists}
% \newcommand{\st}{\strut}
\newcommand{\dps}[1]{\displaystyle{#1}}

\newenvironment{solution}
  {\color{blue}\begin{proof}[Solution]}
  {\end{proof}\color{black}}

% % From M275 "Topology" at SJSU
% \DeclareMathOperator{\id}{id}
% \newcommand{\taking}[1]{\xrightarrow{#1}}
% \newcommand{\inv}{^{-1}}

% % From M170 "Introduction to Graph Theory" at SJSU
% \DeclareMathOperator{\diam}{diam}
% \DeclareMathOperator{\ord}{ord}
% \newcommand{\defeq}{\overset{\mathrm{def}}{=}}

% % From the USAMO .tex files
% \newcommand{\ts}{\textsuperscript}
% \newcommand{\dg}{^\circ}
% \newcommand{\ii}{\item}
