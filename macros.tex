% macros.tex

% General
\let\oldin\in
\let\oldcdot\cdot
\renewcommand*{\in}{\ensuremath{\oldin}}
\renewcommand*{\cdot}{\ensuremath{\oldcdot}}
\newcommand*{\rtt}{\ensuremath{\sqrt{2}}}
\newcommand*{\lt}{\left}
\newcommand*{\rt}{\right}
\newcommand*{\st}{\ensuremath{\ \vert \ }} % such that (set builder)
\newcommand*{\bigst}{\ensuremath{\ \big|\ }} % big such that (set builder)
\newcommand*{\biggst}{\ensuremath{\ \bigg|\ }} % bigg such that (set builder)
\newcommand*{\Bigst}{\ensuremath{\ \Big|\ }} % bigg such that (set builder)
\newcommand*{\imp}{\ensuremath{\implies}}
\newcommand*{\pmi}{\ensuremath{\impliedby}}
\renewcommand*{\iff}{\ensuremath{\Longleftrightarrow\ }}
\newcommand*{\qimp}{\ensuremath{\quad\implies\quad}}
\newcommand*{\qpmi}{\ensuremath{\quad\impliedby\quad}}
\newcommand*{\qiff}{\ensuremath{\quad\iff\quad}}
\newcommand*{\tand}{\text{ and }}
\newcommand*{\tor}{\text{ or }}
\newcommand*{\tst}{\text{ such that }}
\newcommand*{\eps}{\epsilon}
\newcommand*{\veps}{\varepsilon}
\newcommand*{\sq}{^2}
\newcommand*{\cb}{^3}
\newcommand*{\nonzero}[1]{\ensuremath{#1_{\neq 0}}}
\newcommand*{\sseq}{\ensuremath{\subseteq}}
\newcommand{\R}{\ensuremath{\mathbb{R}}} % reals
\newcommand{\N}{\ensuremath{\mathbb{N}}} % naturals
\newcommand{\Z}{\ensuremath{\mathbb{Z}}} % integers
\newcommand{\Q}{\ensuremath{\mathbb{Q}}} % rationals
\newcommand{\C}{\ensuremath{\mathbb{C}}} % complex
\newcommand{\M}{\ensuremath{\mathbb{M}}} % matrices

\newcommand{\boolT}{\tbo{True}}  % bool True
\newcommand{\boolF}{\tbo{False}} % bool False

%%% Misc. operators
\DeclareMathOperator{\sign}{sign} % signum function
\DeclareMathOperator{\Aut}{Aut} % automorphism group
\DeclareMathOperator{\Inn}{Inn} % inner automorphism group
\DeclareMathOperator{\Syl}{Syl} % Sylow subgroup
\DeclareMathOperator{\GL}{GL} % general linear group
\DeclareMathOperator{\SL}{SL} % special linear group

%%% Font variations
\newcommand{\nf}[1]{\textnormal{#1}}  % normal
\newcommand{\tit}[1]{\emph{#1}}       % italics
\newcommand{\tbo}[1]{\textbf{#1}}     % bold
\newcommand{\tun}[1]{\underline{#1}}  % underline
\newcommand{\tmo}[1]{\texttt{#1}}     % monospace/typewriter

%%% Add images to document
\newcommand{\img}[2]{\begin{figure}[H] % parameter 1: image path // parameter 2: height/scale
  \centering%
  \includegraphics[height=#2 \textwidth]{#1}%
  \end{figure}}

% SPRING 2025
% CS 390 ATA Advanced Topics on Algorithms
\newcommand{\floor}[1]{\ensuremath{\lfloor #1 \rfloor}}
\newcommand{\doc}[1]{\noindent\tit{\color{commentgreen} \texttt{/*}\ #1 \hfill \texttt{*/}}}
\newcommand{\rcomm}[1]{\tit{\color{commentgreen} \hfill \texttt{//} #1 }}
\newcommand{\lcomm}[1]{\tit{\color{commentgreen} \texttt{//} #1 }}

% MA 454 Galois Theory Honors
\DeclareMathOperator{\Gal}{Gal} % Galois group

% FALL 2024
%%% CS 314 "Numerical Methods"
\newcommand{\Prob}{\mathbb{P}}
\newcommand{\Exep}{\mathbb{E}}
\newcommand{\asmp}{\textbf{Assumptions: }}
\DeclareMathOperator{\var}{\text{Var}} % variance

%%% MA 425 "Elements of Complex Analysis"
\newcommand{\len}[1]{\ensuremath{\lt\vert #1 \rt\vert}}
\newcommand{\re}{\operatorname{Re}}
\newcommand{\im}{\operatorname{Im}}
\newcommand{\Arg}{\operatorname{Arg}}
\newcommand{\wbar}{\overline{w}}
\renewcommand{\bar}[1]{\overline{#1}}

%%% MA 450 "Honors Abstract Algebra"
\newcommand*{\divs}{\ensuremath{\,\big\vert\,}} % a divides k <=> a \divs k
\newcommand*{\ndivs}{\ensuremath{\,\nmid\,}} % a does not divide k <=> a \ndivs k
\newcommand*{\suth}{\text{such that}}
\newcommand*{\homo}{\text{homomorphism}}
\newcommand{\inv}{\ensuremath{^{-1}}}
\newcommand{\order}[1]{\ensuremath{\lt\vert #1 \rt\vert}} % order of element/group
\newcommand*{\sgp}{\ensuremath{\leq}} % subgroup
\newcommand*{\psgp}{\ensuremath{<}} % proper subgroup
\newcommand*{\tsgp}{\ensuremath{\{e\}}} % trivial subgroup
\newcommand{\cyc}[1]{\ensuremath{\langle #1 \rangle}} % cyclic group
\newcommand*{\nsgp}{\vartriangleleft} % normal subgroup
\newcommand*{\iso}{\cong} % isomorphism
\renewcommand*{\index}[2]{\ensuremath{[ #1 : #2 ]}} % coset index
\newcommand*{\edp}{\oplus} % external direct product
\newcommand*{\idp}{\times} % internal direct product
% \newcommand*{\abar}{\ensuremath{\overline{a}}}
% \newcommand*{\ebar}{\ensuremath{\overline{e}}}
% \newcommand*{\xbar}{\ensuremath{\bar{x}}}
% \newcommand*{\Gbar}{\ensuremath{\overline{G}}}
% \newcommand*{\Hbar}{\ensuremath{\overline{H}}}
% \newcommand*{\Kbar}{\ensuremath{\overline{K}}}
\newcommand*{\contradiction}{\ensuremath{(\Rightarrow\!\Leftarrow)}} % contradiction
\newcommand{\U}[1]{U(#1)} % group of units modulo n under multiplication
\newcommand{\ZZ}[1]{\ensuremath{\Z_{#1}}} % group of nonnegative integers < input under addition
\newcommand*{\Zn}{\ensuremath{\Z_n}}      % group of nonnegative integers < n under addition
\newcommand*{\Ztw}{\ensuremath{\Z_2}}     % group of nonnegative integers < 2 under addition
\newcommand*{\Zth}{\ensuremath{\Z_3}}     % group of nonnegative integers < 3 under addition
\newcommand*{\Zfo}{\ensuremath{\Z_4}}     % group of nonnegative integers < 4 under addition
\newcommand*{\Zfi}{\ensuremath{\Z_5}}     % group of nonnegative integers < 5 under addition
\newcommand*{\Zsi}{\ensuremath{\Z_6}}     % group of nonnegative integers < 6 under addition
\newcommand*{\Zse}{\ensuremath{\Z_7}}     % group of nonnegative integers < 7 under addition
\newcommand*{\Zei}{\ensuremath{\Z_8}}     % group of nonnegative integers < 8 under addition
\newcommand*{\Zni}{\ensuremath{\Z_9}}     % group of nonnegative integers < 9 under addition
\newcommand*{\Zte}{\ensuremath{\Z_{10}}}     % group of nonnegative integers less than 9 under addition
\newcommand{\orb}[2]{\ensuremath{\operatorname{orb}_{#1}(#2)}} % orbit
\newcommand{\stab}[2]{\ensuremath{\operatorname{stab}_{#1}(#2)}} % stabilizer
\newcommand{\qg}[2]{\ensuremath{#1\,/\,#2}} % quotient group
\newcommand*{\conj}[2]{\ensuremath{#1 #2 #1\inv}} % conjugate
\newcommand{\cl}[1]{\ensuremath{\operatorname{cl}( #1 )}} % conjugate class
\renewcommand*{\char}[1]{\ensuremath{\operatorname{char}(#1)}} % characteristic
\newcommand{\defmap}[5]{\ensuremath{#1:\begin{matrix}
    #2 \to #3 \\
    #4 \mapsto #5
\end{matrix}}}
\newcommand*{\srg}{\ensuremath{\leq}} % subring
\newcommand*{\psrg}{\ensuremath{<}} % proper subring
\newcommand*{\tsrg}{\ensuremath{\{0\}}} % trivial subring
\newcommand*{\idl}{\vartriangleleft} % ideal
\newcommand{\pidl}[1]{\ensuremath{\langle #1 \rangle}} % principal ideal
\newcommand*{\notidl}{\ntriangleleft} % not ideal
\newcommand{\fr}[2]{\ensuremath{#1\,/\,#2}} % factor ring

% SUMMER 2024
%%% STAT 350 "Introduction to Statistics"
\newcommand{\dint}[2]{\int_{#1}^{#2}}
\newcommand{\Dint}{\int_{-\infty}^{\infty}}

% SPRING 2024
%%% MA 341 "Foundations of Analysis"
\newcommand{\abs}[1]{\ensuremath{\lt\vert #1 \rt\vert}}

%%% MA 35301 "Linear Algebra II"
\newcommand{\inner}[2]{\ensuremath{\lt\langle #1, #2 \rt\rangle}}
\newcommand{\norm}[1]{\ensuremath{\lt\vert\lt\vert #1 \rt\vert\rt\vert}}

% FALL 2023
%%% MA 375 "Introduction to Discrete Mathematics"
\newcommand{\Mod}[1]{\ \mathrm{mod}\ #1}


% From Math 55 and Math 145 at Harvard
\newenvironment{subproof}[1][Proof]{%
\begin{proof}[#1] \renewcommand{\qedsymbol}{$\blacksquare$}}%
{\end{proof}}


\newcommand{\ol}{\overline}
% \newcommand{\tun}{\underline}
\newcommand{\wt}{\widetilde}
\newcommand{\wh}{\widehat}
\newcommand{\vocab}[1]{\textbf{\color{blue} #1}}
% \newcommand{\exercise}[1]{\textbf{\color{red} #1}}
\providecommand{\half}{\frac{1}{2}}
% \newcommand{\dang}{\measuredangle} %% Directed angle
% \newcommand{\ray}[1]{\overrightarrow{#1}}
% \newcommand{\seg}[1]{\overline{#1}}
% \newcommand{\arc}[1]{\wideparen{#1}}
% \DeclareMathOperator{\cis}{cis}
% \DeclareMathOperator*{\argmin}{arg min}
% \DeclareMathOperator*{\argmax}{arg max}
% \newcommand{\cycsum}{\sum_{\mathrm{cyc}}}
% \newcommand{\symsum}{\sum_{\mathrm{sym}}}
% \newcommand{\cycprod}{\prod_{\mathrm{cyc}}}
% \newcommand{\symprod}{\prod_{\mathrm{sym}}}
% \newcommand{\Qed}{\begin{flushright}\qed\end{flushright}}
% \newcommand{\parinn}{\setlength{\parindent}{1cm}}
% \newcommand{\parinf}{\setlength{\parindent}{0cm}}
% \newcommand{\norm}{\|\cdot \|}
% \newcommand{\inorm}{\norm_{\infty}}
% \newcommand{\opensets}{\{V_{\alpha}\}_{\alpha\in I}}
% \newcommand{\oset}{V_{\alpha}}
% \newcommand{\opset}[1]{V_{\alpha_{#1}}}
% \newcommand{\lub}{\text{lub}}
% \newcommand{\del}[2]{\frac{\partial #1}{\partial #2}}
% \newcommand{\Del}[3]{\frac{\partial^{#1} #2}{\partial^{#1} #3}}
% \newcommand{\deld}[2]{\dfrac{\partial #1}{\partial #2}}
% \newcommand{\Deld}[3]{\dfrac{\partial^{#1} #2}{\partial^{#1} #3}}
% \newcommand{\lm}{\lambda}
% \newcommand{\uin}{\mathbin{\rotatebox[origin=c]{90}{$\in$}}}
% \newcommand{\usubset}{\mathbin{\rotatebox[origin=c]{90}{$\subset$}}}
% \newcommand{\lt}{\left}
% \newcommand{\rt}{\right}
% \newcommand{\bs}[1]{\boldsymbol{#1}}
% \newcommand{\exs}{\exists}
% \newcommand{\st}{\strut}
% \newcommand{\dps}[1]{\displaystyle{#1}}


%%% From M275 "Topology" at SJSU
% \DeclareMathOperator{\id}{id}
% \newcommand{\taking}[1]{\xrightarrow{#1}}
% \newcommand{\inv}{^{-1}}

%%% From M170 "Introduction to Graph Theory" at SJSU
% \DeclareMathOperator{\diam}{diam}
% \DeclareMathOperator{\ord}{ord}
% \newcommand{\defeq}{\overset{\mathrm{def}}{=}}

%%% From the USAMO .tex files
% \newcommand{\ts}{\textsuperscript}
% \newcommand{\dg}{^\circ}
% \newcommand{\ii}{\item}
