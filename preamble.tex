% preamble.sty
% \ProvidesPackage{preamble}

% Inspired by http://www.mit.edu/~lindrew/


\usepackage[utf8]{inputenc}
%% Language and font encodings
\usepackage[english]{babel}
\usepackage[T1]{fontenc}
%% Sets page size and margins
\usepackage[a4paper,top=3cm,bottom=2cm,left=0.8in,right=0.8in,marginparwidth=1.75cm]{geometry} % jakin

%% Useful packages
\usepackage{amsmath, amsthm, amssymb} %standard
\usepackage{graphicx} %include images/graphics
\usepackage{dsfont} %includes some symbols?
\usepackage{systeme} %system of equations
\usepackage{enumitem} %bullet points
\usepackage{tikz} %draw diagrams
\usepackage{fancyhdr} \pagestyle{fancy} %add headers
    \rhead{\thepage} %right head is page number
    \lhead{\leftmark} %left head is the section name
\usepackage{mathtools}
\usepackage{multicol}
\usepackage{xparse}
\usepackage{bbm} %blackboard fonts
\usepackage{etoolbox}
\usepackage{xcolor}
\usepackage[activate={true,nocompatibility},final,tracking=true,kerning=true,spacing=true,factor=1100,stretch=10,shrink=10]{microtype} \microtypecontext{spacing=nonfrench} %adjusts spacing
\usepackage{centernot}
\usepackage{parskip} %skips the paragraph indent?
\usepackage{stmaryrd} %CS font
\usepackage{esvect} %lets me use \vv for a vector arrow
\usepackage[colorlinks=true, allcolors=blue]{hyperref} %makes links blue

% ethan's packages:
\usepackage{float}

% new commands
\newcommand{\by}{\hspace{0.05em}{\times}\hspace{0.05em}}
\newcommand{\id}{\text{id}}

% Blackboard bold
\newcommand{\CC}{\mathbb C}
% \newcommand{\EE}{\mathbb E}
\newcommand{\FF}{\mathbb F}
\newcommand{\NN}{\mathbb N}
\newcommand{\QQ}{\mathbb Q}
\newcommand{\RR}{\mathbb R}
% \newcommand{\ZZ}{\mathbb Z}
\newcommand{\HH}{\mathbb H}

%18.701
%\newcommand{\ord}{\text{ord}}
\DeclareMathOperator{\ord}{ord}
% \newcommand{\im}{\text{im}}
\DeclareMathOperator{\im}{im}
\newcommand{\nsub}{\trianglelefteq}
\newcommand{\rto}{\longrightarrow}
\newcommand{\vvv}{\vv{v}}
\newcommand{\mto}{\longmapsto}
\newcommand{\spann}{\text{Span}}
\newcommand{\matnn}{\text{Mat}_{n \by n}}

% \newcommand{\stab}{\text{Stab}}
% \newcommand{\orbit}{\text{Orbit}}
% \newcommand{\perm}{\text{Perm}}
% \DeclareMathOperator{\stab}{Stab}
\DeclareMathOperator{\orbit}{Orbit}
\DeclareMathOperator{\perm}{Perm}
\DeclareMathOperator{\proj}{proj}
% \DeclareMathOperator{\Ker}{ker}
% \DeclareMathOperator{\GL}{GL}
\DeclareMathOperator{\trace}{Tr}
% \newcommand{\bform}{\langle \cdot, \cdot \rangle}
\DeclareMathOperator{\lie}{Lie}


\DeclarePairedDelimiter\ceil{\lceil}{\rceil}
\DeclarePairedDelimiter\floor{\lfloor}{\rfloor}

\usepackage{thmtools}
\usepackage[framemethod = TikZ]{mdframed}
\usepackage{tikz-cd}

\usepackage{silence} % for suppressing warnings (thanks, Jason)
\WarningFilter{mdframed}{You got a bad break}
\mdfsetup{skipabove=1em,skipbelow=0em} %puts space before and after

\mdfsetup{
	linewidth = 0.3mm,
	innertopmargin = 2mm,
	innerbottommargin = 3.5mm,
	innerleftmargin = 3mm,
	innerrightmargin = 3mm
} % adjusts boundaries of boxes

\newcommand{\thmboxstyle}[4]{
	\mdfdefinestyle{#2}{
		linecolor = #3,
		backgroundcolor = #4,
		nobreak = true
	}
	\declaretheoremstyle[
		headfont = \bfseries\color{#3},
		mdframed = {style = #2},
		headpunct = {.},
		postheadspace = {5pt},
	]{#1}
}

\definecolor{oran}{HTML}{877AC6}
\definecolor{redd}{rgb}{0.6,0,0}

% four different colors of boxes
\thmboxstyle{defbox}{mdgreenbox}{green!70!black}{green!10!white} % green; definition box
\thmboxstyle{thmbox}{mdredbox}{redd}{blue!30!red!10} % red; theorem box
\thmboxstyle{exbox}{mdbluebox}{blue!110!red!66!green!66}{blue!5} % cyan; example box
\thmboxstyle{ntbox}{mdorangebox}{orange!50!brown}{yellow!5!olive!5} % orange; note box
\thmboxstyle{exrbox}{mdpurplebox}{pink!70!red}{pink!10!white} % purple; exercise box

\declaretheorem[style = thmbox, name = Theorem, numberwithin = section]{theorem}
\declaretheorem[style = thmbox, name = Lemma, numberwithin = section]{lemma}
\declaretheorem[style = thmbox, name = Proposition, numberwithin = section]{proposition}
\declaretheorem[style = thmbox, name = Corollary, numberwithin = theorem]{corollary}
\declaretheorem[style = thmbox, name = Conjecture, numberwithin = section]{conjecture}
\declaretheorem[style = thmbox, name = Algorithm, numberwithin = section]{algorithm}

\declaretheorem[style = thmbox, name = Theorem, numbered = no]{theorem*}
\declaretheorem[style = thmbox, name = Lemma, numbered = no]{lemma*}
\declaretheorem[style = thmbox, name = Proposition, numbered = no]{proposition*}
\declaretheorem[style = thmbox, name = Corollary, numbered = no]{corollary*}
\declaretheorem[style = thmbox, name = Conjecture, numbered = no]{conjecture*}
\declaretheorem[style = thmbox, name = Algorithm, numbered = no]{algorithm*}

\declaretheorem[style = defbox, name = Definition, numberwithin = section]{definition}
\declaretheorem[style = defbox, name = Definition, numbered = no]{definition*}

\declaretheorem[style = exbox, name = Example, numberwithin = section]{example}
\declaretheorem[style = exbox, name = Example, numbered = no]{example*}
\declaretheorem[style = exbox, name = Non-Example, numberwithin = section]{nonexample}
\declaretheorem[style = exbox, name = Non-Example, numbered = no]{nonexample*}

\declaretheorem[style = exbox, name = Examples, numbered = no]{examples}
\declaretheorem[style = exbox, name = Non-Examples, numbered = no]{nonexamples}

\declaretheorem[style = ntbox, name = Remark, numbered = no]{remark}
\declaretheorem[style = ntbox, name = Fact, numbered = no]{fact}
\declaretheorem[style = ntbox, name = Note, numbered = no]{note}

\declaretheorem[style = plain, name = Question, numbered = no]{ques}
\declaretheorem[style = plain, name = Answer, numbered = no]{ans}
\declaretheorem[style = ntbox, name = Problem]{problem}
\declaretheorem[style = ntbox, name = Problem, numbered = no]{problem*}
\declaretheorem[style = ntbox, name = Resources, numbered = no]{resources}
\declaretheorem[style = ntbox, name = Warning, numbered = no]{warning}
\declaretheorem[style = exrbox, name = Exercise, numbered = no]{exercise}

% \declaretheorem[style = plain, name = Student Question, numbered = no]{question}
\declaretheorem[style = plain, name = Question, numbered = no]{question*}

\declaretheorem[style = plain, name = Claim]{claim}
\declaretheorem[style = plain, name = Claim, numbered = no]{claim*}

\def\todo#1{\relax}

% >>-----------------------------------| park |-----------------------------------<<

% Adjusts space before and after subsections
\usepackage{titlesec}
\titlespacing*{\subsection}{0pt}{3.5ex plus 1ex minus .2ex}{3.5ex plus .2ex}

% >>------------------------------------------------------------------------------<<
